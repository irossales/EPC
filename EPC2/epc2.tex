\documentclass{report}
\usepackage[T1]{fontenc}
\usepackage[utf8]{inputenc}
\usepackage{lmodern}
\usepackage{hyperref}
\usepackage[portuges,brazilian]{babel}
\usepackage{graphicx}
\usepackage{textcomp}
\usepackage{fullpage}
\usepackage{wrapfig}
\usepackage{float}
\usepackage{listings}
\usepackage{amsmath}
\usepackage{amssymb}
\begin{document}

\newcommand{\HRule}{\rule{\linewidth}{0.5mm}}
\newcommand{\tsize}[1]{(\frac{W}{L})_{#1}}
 

%%%%%%%%%%%%%%%%%%%%%%%%%% START TITLE PAGE %%%%%%%%%%%%%%%%%%%%%%%%5
\begin{titlepage}

\begin{center}


{\LARGE UNIVERSIDADE DE SÃO PAULO\\}
{\LARGE DEPARTAMENTO DE ENGENHARIA ELÉTRICA \\}
{\LARGE ESCOLA DE ENGENHARIA DE SÃO CARLOS\\[4cm]}

\textbf{\large SEL5755 - Sistemas Fuzzy}\\[1cm]
\textbf{\large Prof Dr. Ivan Nunes da Silva}\\[2cm]


% Title
\HRule \\[0.6cm]
{ \huge EPC 2\bfseries }\\[0.6cm]

\HRule \\[2cm]

% Author

\begin{center} \large
\emph{Alunos:}\\
\end{center}

\begin{minipage}{0.4\textwidth}
\begin{flushleft} \large
Isabela R. do Prado \textsc{Rossales}\\
6445435
\end{flushleft}
\end{minipage}
\begin{minipage}{0.4\textwidth}
\begin{flushright} \large
Jonas Rossi \textsc{Dourado}\\
6445442
\end{flushright}
\end{minipage}

\vfill

% Bottom of the page
{\large São Carlos,\\ \today}

\end{center}

\end{titlepage}
%\listoffigures
%\begingroup
%\let\clearpage\relax
%\listoftables
%\endgroup
%%%%%%%%%%%%%%%%%%%%%%%%%% STOP TITLE PAGE %%%%%%%%%%%%%%%%%%%%%%%%5


\newpage

\begin{enumerate}

\item[1] Considere o conjunto \emph{fuzzy} A definido no universo de discurso $X = \{ x \in  \mathbb{R} \vert 0 \le x \le 10\}$,
o qual é representado pela seguinte função de pertinência:


\begin{equation*}
\mu_A (x) = 
\begin{cases} 
0,5x-1,5, & \text{se $3 \leq x \leq 5$}
\\
-0,5x+3,5, & \text{se $5 < x \leq 7$}
\\
0, &\text{caso contrário}
\end{cases}
\end{equation*}

\begin{enumerate}
    \item[a)] Esboce o gráfico da função de pertinência representada acima, indicando também qual é o seu tipo.
    \item[b)] Sabendo-se que a função acima está representando o conjunto referente à temperatura ``média'' de um determinado
    processo industrial, explique então qual o significado que está embutido em tal representação.
    \item[c)] Explique se o conjunto \emph{fuzzy} acima é considerado um conjunto normalizado.
    \item[d)] Obtenha o conjunto suporte associado ao conjunto \emph{fuzzy} acima.
\end{enumerate}

\item[2] Calcule a cardinalidade dos conjuntos \emph{fuzzy} discretos dados a seguir:
\begin{enumerate}
    \item[a)] $A = 0,3/x_1 + 0,5/x_2 + 0,9/x_3 + 0,4/x_4 + 0,1/x_5$
    \item[b)] $A = 0,0/x_1 + 0,4/x_2 + 1,0/x_3 + 1,0/x_4 + 0,4/x_5 + 0,0/x_6$
    \item[c)] $\mu_c(x)=\frac{x}{x+1}, \text{com } x \in \{0,1,2,...,10 \}$
\end{enumerate}


\end{enumerate}


\end{document}
