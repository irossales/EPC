\documentclass{report}
\usepackage[T1]{fontenc}
\usepackage[utf8]{inputenc}
\usepackage{lmodern}
\usepackage{hyperref}
\usepackage[portuges,brazilian]{babel}
\usepackage{graphicx}
\usepackage{textcomp}
\usepackage{fullpage}
\usepackage{wrapfig}
\usepackage{float}
\usepackage{listings}
\usepackage{amsmath}
\begin{document}

\newcommand{\HRule}{\rule{\linewidth}{0.5mm}}
\newcommand{\tsize}[1]{(\frac{W}{L})_{#1}}
 

%%%%%%%%%%%%%%%%%%%%%%%%%% START TITLE PAGE %%%%%%%%%%%%%%%%%%%%%%%%5
\begin{titlepage}

\begin{center}


{\LARGE UNIVERSIDADE DE SÃO PAULO\\}
{\LARGE DEPARTAMENTO DE ENGENHARIA ELÉTRICA \\}
{\LARGE ESCOLA DE ENGENHARIA DE SÃO CARLOS\\[4cm]}

\textbf{\large SEL5755 - Sistemas Fuzzy}\\[1cm]
\textbf{\large Prof Dr. Ivan Nunes da Silva}\\[2cm]


% Title
\HRule \\[0.6cm]
{ \huge EPC 1\bfseries }\\[0.6cm]

\HRule \\[2cm]

% Author

\begin{center} \large
\emph{Alunos:}\\
\end{center}

\begin{minipage}{0.4\textwidth}
\begin{flushleft} \large
Isabela R. do Prado \textsc{Rossales}\\
6445435
\end{flushleft}
\end{minipage}
\begin{minipage}{0.4\textwidth}
\begin{flushright} \large
Jonas Rossi \textsc{Dourado}\\
6445442
\end{flushright}
\end{minipage}

\vfill

% Bottom of the page
{\large São Carlos,\\ \today}

\end{center}

\end{titlepage}
%\listoffigures
%\begingroup
%\let\clearpage\relax
%\listoftables
%\endgroup
%%%%%%%%%%%%%%%%%%%%%%%%%% STOP TITLE PAGE %%%%%%%%%%%%%%%%%%%%%%%%5


\newpage

\begin{enumerate}

\item[1] Considere os conjuntos A e B definidos como se segue, os quais modelam intervalos sobre o segmento real $\Re$.

$A = \{x \in \Re | -1 \leq x \leq 2\}$

$B = \{x \in \Re |  1 \leq x \leq 4\}$ 
\begin{enumerate}

\item[i.] Encontre a função característica de A e B.
%-------------------------------------------------------------------------------------------------

\begin{equation*}
\mu_A (x) = 
\begin{cases} 
1, & \text{se $-1 \leq x \leq 2$}
\\
0, &\text{caso contrário}
\end{cases}
\end{equation*}

\begin{equation*}
\mu_B (x) = 
\begin{cases} 
1, & \text{se $1 \leq x \leq 4$}
\\
0, &\text{caso contrário}
\end{cases}
\end{equation*}


%-------------------------------------------------------------------------------------------------
\item[ii.] Determine a união, a interseção, e o complemento dos conjuntos A e B em termos de suas funções características.
\end{enumerate}
%-------------------------------------------------------------------------------------------------

$\mu_{A \cup B} = max \lbrace\mu_A, \mu_B\rbrace$

$\mu_{A \cap B} = min \lbrace \mu_A, \mu_B   \rbrace  = \mu_A \mu_B $

$\mu_{A - B}    = max \lbrace \mu_A - \mu_B, 0 \rbrace   $

$\mu_{B - A}    = max \lbrace \mu_B - \mu_A, 0 \rbrace   $

$A \cup B = \lbrace x \vert \mu_{A \cup B }(x) = 1 \rbrace$

$A \cap B = \lbrace x \vert \mu_{A \cap B }(x) = 1 \rbrace$

$A - B = \lbrace x \vert \mu_{A - B }(x) = 1 \rbrace$

$B - A = \lbrace x \vert \mu_{B - A }(x) = 1 \rbrace$

%-------------------------------------------------------------------------------------------------
\item[2] Encontre a função característica do conjunto $E = \{ x | x \neq x \}$.
%-------------------------------------------------------------------------------------------------

$ \mu_E(x) = 0$ 

%-------------------------------------------------------------------------------------------------
\item[3] Demonstre as leis de De Morgan, ou seja:

\begin{enumerate}
    \item[i.] $\overline{A \cap B} = \overline{A} \cup \overline{B}$

    $\mu_{\overline{A \cap B}} = 1 - \mu_{A \cap B} $
    
\hspace{0.9 cm}    $ = 1 - \mu_A \mu_B$

 \hspace{0.9 cm}   $ = (1 - \mu_A) + (1 - \mu_B) - (1 - \mu_A)(1-\mu_B)$

 \hspace{0.9 cm}   $ = \mu_{\overline{A}} + \mu_{\overline{B}} - \mu_{\overline{A}\cap\overline{B}}$

\hspace{0.9 cm}  $ = \mu_{\overline{A}\cup\overline{B}} $ 

    \item[ii.] $\overline{A \cup B} = \overline{A} \cap \overline{B}$

\end{enumerate}


\item[4] Mostrar mediante diagramas apropriados que:

$A \cup (B \cap C) = (A \cup B) \cap (A \cup C)$

\item[5] Do ponto de vista da engenharia (ou da computação), explique os seguintes aspectos:

\begin{enumerate}
\item[i.] Como você definiria para um profissional leigo o que é um sistema Fuzzy.

Sistemas Fuzzy são sistemas que utilizam de informações imprecisas para tomar decisões, sendo semelhantes 
ao modelo de tomada de decisões dos seres humanos. Por exemplo, uma pessoa pode achar que está ``um pouco
frio'' e então procurar se agasalhar. O sistema de decisão adotado por essa pessoa é um modelo Fuzzy. 

Note que dessa maneira uma variável (no caso, a temperatura ambiente) 
pode pertencer a mais de uma classe com diferentes níveis de verdade. Uma pessoa no
mesmo ambiente que esteja praticando atividade física possivelmente falaria que a sensação térmica é ``normal'' ou mesmo
``calor''.

\item[ii.] Explicite um exemplo de aplicação em que poderia ser utilizado um sistema Fuzzy, 
enumerando as características das variáveis de entrada e de saída.

Um exemplo de uma aplicação é um sistema que estima qual a chance de uma pessoa que não tem diabetes tipo 2
desenvolver a doença no futuro de acordo com o estilo de vida. 

Entre as entradas podemos citar, por exemplo, quantidade 
de exercício por dia (pouco, normal, muito), consumo de doces (pouco, razoável, muito), peso (magro, normal, obeso) e 
estatura (baixa, normal, alta).  Como variável de saída, a probabilidade de desenvolver diabetes (pequena, média, alta).

\item[iii.] Discorra quando um determinado tipo de problema se torna qualificável para 
resolvê-lo pela aplicação de sistemas Fuzzy.

Um problema se torna qualificável para ser resolvido utilizando sistemas fuzzy quando as variáveis de entrada tratadas pelo sistema
descrevem imprecisão ou quando há conceitos subjetivos envolvidos. Ex.: um sistema que determina se uma pessoa apresentará problemas
de saúde baseado no consumo de frutas, água etc., consumo esse descrito em termos qualitativos (pouco, médio, muito).

Quando a saída do sistema não possuir uma definição
exata e quantitativa, ele também é mais apropriado para ser descrito como um sistema fuzzy.
Ex.: um sistema de controle de freios de modo que a parada do carro não seja muito brusca (saída: confortabilidade
do passageiro durante a frenagem do carro).

Além disso, certos problemas passíveis de serem resolvidos por sistemas exatos, mas com um conjunto de variáveis
de entrada muito grande ou que necessitem de diversos sensores para uma captação quantitativa dos dados,
podem ser resolvidos de forma menos complexa por sistemas fuzzy.

\end{enumerate}

\end{enumerate}


\end{document}
