\documentclass{report}
\usepackage[T1]{fontenc}
\usepackage[utf8]{inputenc}
\usepackage{lmodern}
\usepackage{hyperref}
\usepackage[portuges,brazilian]{babel}
\usepackage{graphicx}
\usepackage{textcomp}
\usepackage{fullpage}
\usepackage{wrapfig}
\usepackage{float}
\usepackage{listings}
\begin{document}

\newcommand{\HRule}{\rule{\linewidth}{0.5mm}}
\newcommand{\tsize}[1]{(\frac{W}{L})_{#1}}
 

%%%%%%%%%%%%%%%%%%%%%%%%%% START TITLE PAGE %%%%%%%%%%%%%%%%%%%%%%%%5
\begin{titlepage}

\begin{center}


{\LARGE UNIVERSIDADE DE SÃO PAULO\\}
{\LARGE DEPARTAMENTO DE ENGENHARIA ELÉTRICA \\}
{\LARGE ESCOLA DE ENGENHARIA DE SÃO CARLOS\\[4cm]}

\textbf{\large SEL5755 - Sistemas Fuzzy}\\[1cm]
\textbf{\large Prof Dr. Ivan Nunes da Silva}\\[2cm]


% Title
\HRule \\[0.6cm]
{ \huge EPC 1\bfseries }\\[0.6cm]

\HRule \\[2cm]

% Author

\begin{center} \large
\emph{Alunos:}\\
\end{center}

\begin{minipage}{0.4\textwidth}
\begin{flushleft} \large
Isabela R. do Prado \textsc{Rossales}\\
6445435
\end{flushleft}
\end{minipage}
\begin{minipage}{0.4\textwidth}
\begin{flushright} \large
Jonas Rossi \textsc{Dourado}\\
6445442
\end{flushright}
\end{minipage}

\vfill

% Bottom of the page
{\large São Carlos,\\ \today}

\end{center}

\end{titlepage}
%\listoffigures
%\begingroup
%\let\clearpage\relax
%\listoftables
%\endgroup
%%%%%%%%%%%%%%%%%%%%%%%%%% STOP TITLE PAGE %%%%%%%%%%%%%%%%%%%%%%%%5


\newpage

\begin{enumerate}

\item[1] Considere os conjuntos A e B definidos como se segue, os quais modelam intervalos sobre o segmento real $\Re$.

$A = \{x \in \Re | -1 \leq x \leq 2\}$

$B = \{x \in \Re |  1 \leq x \leq 4\}$ 
\begin{enumerate}
\item[i.] Encontre a função característica de A e B.

\item[ii.] Determine a união, a interseção, e o complemento dos conjuntos A e B em termos de suas funções características.
\end{enumerate}

\item[2] Encontre a função característica do conjunto $E = \{ x | x \neq x \}$.


\item[3] Demonstre as leis de De Morgan, ou seja:

\begin{enumerate}
    \item[i.] $\overline{A \cap B} = \overline{A} \cup \overline{B}$

    \item[ii.] $\overline{A \cup B} = \overline{A} \cap \overline{B}$
\end{enumerate}


\item[4] Mostrar mediante diagramas apropriados que:

$A \cup (B \cap C) = (A \cup B) \cap (A \cup C)$

\item[5] Do ponto de vista da engenharia (ou da computação), explique os seguintes aspectos:

\begin{enumerate}
\item[i.] Como você definiria para um profissional leigo o que é um sistema fuzzy.

\item[ii.] Explicite um exemplo de aplicação em que poderia ser utilizado um sistema fuzzy, 
enumerando as características das variáveis de entrada e de saída.

\item[iii.] Discorra quando um determinado tipo de problema se torna qualificável para 
resolvê-lo pela aplicação de sistemas fuzzy.
\end{enumerate}

\end{enumerate}


\end{document}
