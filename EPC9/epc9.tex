\documentclass{report}
\usepackage[T1]{fontenc}
\usepackage[utf8]{inputenc}
\usepackage{lmodern}
%\usepackage{hyperref}
\usepackage[portuges,brazilian]{babel}
\usepackage{graphicx, subfigure}
\usepackage{textcomp}
\usepackage{fullpage}
\usepackage{wrapfig}
\usepackage{float}
\usepackage{listings}
\usepackage{amsmath}
\usepackage{amssymb}
\usepackage[margin=0.5in]{geometry}
\usepackage{pdfpages}

\begin{document}

\newcommand{\HRule}{\rule{\linewidth}{0.5mm}}
\newcommand{\tsize}[1]{(\frac{W}{L})_{#1}}
 

%%%%%%%%%%%%%%%%%%%%%%%%%% START TITLE PAGE %%%%%%%%%%%%%%%%%%%%%%%%5
\begin{titlepage}

\begin{center}


{\LARGE UNIVERSIDADE DE SÃO PAULO\\}
{\LARGE DEPARTAMENTO DE ENGENHARIA ELÉTRICA \\}
{\LARGE ESCOLA DE ENGENHARIA DE SÃO CARLOS\\[4cm]}

\textbf{\large SEL5755 - Sistemas Fuzzy}\\[1cm]
\textbf{\large Prof Dr. Ivan Nunes da Silva}\\[2cm]


% Title
\HRule \\[0.6cm]
{ \huge EPC 9\bfseries }\\[0.6cm]

\HRule \\[2cm]

% Author

\begin{center} \large
\end{center}

\begin{minipage}{\textwidth}
\begin{flushleft} \large
Isabela R. do Prado \textsc{Rossales}\\
6445435
\end{flushleft}
\end{minipage}

\vfill

% Bottom of the page
{\large São Carlos,\\ \today}

\end{center}

\end{titlepage}
%\listoffigures
%\begingroup
%\let\clearpage\relax
%\listoftables
%\endgroup
%%%%%%%%%%%%%%%%%%%%%%%%%% STOP TITLE PAGE %%%%%%%%%%%%%%%%%%%%%%%%5


\newpage

\includepdf[pages=1]{EPC09_Fuzzy.pdf}
\includepdf[pages=2]{EPC09_Fuzzy.pdf}


\begin{enumerate}
\item[1.]
    A seguir encontram-se as funções de regressão:
Regra 1: Valores(d,a,b,c) para f = a*x1 + b*x2 + c*x3 + d): [ 0.07288626  0.35329376  0.34483924  0.46038057] \\
Regra 2: Valores(d,a,b,c) para f = a*x1 + b*x2 + c*x3 + d): [ 0.08024507  0.43866452  0.37704934  0.34253576] \\
Regra 3: Valores(d,a,b,c) para f = a*x1 + b*x2 + c*x3 + d): [ 0.08576458  0.44031065  0.32608057  0.3669811 ] \\
Regra 4: Valores(d,a,b,c) para f = a*x1 + b*x2 + c*x3 + d): [ 0.08248229  0.49501087  0.3198011   0.34368151] \\
Regra 5: Valores(d,a,b,c) para f = a*x1 + b*x2 + c*x3 + d): [ 0.02553684  0.41387792  0.37903706  0.50861781] \\
Regra 6: Valores(d,a,b,c) para f = a*x1 + b*x2 + c*x3 + d): [ 0.07768514  0.44502256  0.42353619  0.31139994] \\
Regra 7: Valores(d,a,b,c) para f = a*x1 + b*x2 + c*x3 + d): [ 0.07609611  0.41948343  0.338711    0.38855778] \\
Regra 8: Valores(d,a,b,c) para f = a*x1 + b*x2 + c*x3 + d): [ 0.13746759  0.48807292  0.26222153  0.30703768] \\
Regra 9: Valores(d,a,b,c) para f = a*x1 + b*x2 + c*x3 + d): [ 0.13746759  0.48807292  0.26222153  0.30703768] \\
Regra 10: Valores(d,a,b,c) para f = a*x1 + b*x2 + c*x3 + d): [ 0.13746759  0.48807292  0.26222153  0.30703768] \\
Regra 11: Valores(d,a,b,c) para f = a*x1 + b*x2 + c*x3 + d): [ 0.13746759  0.48807292  0.26222153  0.30703768] \\
Regra 12: Valores(d,a,b,c) para f = a*x1 + b*x2 + c*x3 + d): [ 0.13746759  0.48807292  0.26222153  0.30703768] \\

\item[4.]
 A seguir Tabela \ref{tab:1} mostra as classes atribuídas pelo sistema.

\begin{table}[h!]
\begin{center}
\caption{Valores de entrada e saída do sistema \emph{fuzzy}.}
\begin{tabular}{c|c|c|c|c}
\textbf{$x_1$} & \textbf{$x_2$} & \textbf{$x_3$} & \textbf{Valor \emph{fuzzy}} &\textbf{Classe atribuída}\\
0.3395& 0.0022& 0.0087& 0.1850 & A \\ 
0.5102& 0.7464& 0.0860& 0.5841 & C \\ 
0.7382& 0.2647& 0.1916& 0.6021 & C \\ 
0.0029& 0.3264& 0.2476& 0.2962 & B \\ 
0.1399& 0.1610& 0.2477& 0.2879 & B \\ 
0.9430& 0.4476& 0.2648& 0.7964 & D \\ 
0.8401& 0.4490& 0.2719& 0.7487 & C \\ 
0.7088& 0.9342& 0.2763& 0.8086 & D \\
0.1957& 0.8423& 0.3085& 0.5573 & C \\
0.0004& 0.6916& 0.5006& 0.4767 & B \\ 
0.6492& 0.0007& 0.6422& 0.6091 & C \\ 
0.1283& 0.1882& 0.7253& 0.4548 & B \\ 
0.2299& 0.1524& 0.7353& 0.4860 & B \\ 
0.0611& 0.2860& 0.7464& 0.4706 & B \\ 
0.1903& 0.6523& 0.7820& 0.6554 & C \\ 
0.2225& 0.9182& 0.7820& 0.7481 & C \\ 
0.6423& 0.3229& 0.8567& 0.7833 & D \\ 
0.3879& 0.1307& 0.8656& 0.5841 & C \\
0.8882& 0.3077& 0.8931& 0.9259 & D \\ 
0.1818& 0.5078& 0.9046& 0.6506 & C \\ 
\end{tabular} 
\label{tab:1}
\end{center}
\end{table}

\item[5.]
Caso a etapa de pré-processamento seja alterada, as funções de regressão serão alteradas e consequentemente
a classificação final. Igualmente, se a etapa de pós-processamento for alterada, a classe atribuída
será influenciada de forma direta.

\end{enumerate}

\cleardoublepage
\lstset{basicstyle=\scriptsize}
\lstinputlisting[language=Python]{epc9.py}
%\lstinputlisting{resp.txt}

\end{document}
